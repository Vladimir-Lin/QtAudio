\documentclass[a4paper,notitlepage,UTF8]{ctexart}
% \documentclass[a4paper,twocolumn,notitlepage,UTF8]{ctexart}

  \usepackage[left=2cm,right=2cm,top=2.5cm,bottom=2.5cm]{geometry}
  \usepackage{amsmath}
  \usepackage{amssymb}
  \usepackage{graphicx}
  \usepackage{latexsym}
  \usepackage{enumerate}
  \title{波$\rightleftharpoons$本相$\rightleftharpoons$粒子}
  \author{Владимир Лесной}
  \date{\today}

\begin{document}
  \maketitle

% \begin{abstract}
% \end{abstract}

\section{波$\rightleftharpoons$本相$\rightleftharpoons$粒子}

本相,介於波與粒子之間的狀態。英文為{\it Original Phase},簡寫為{\it Oriphase}。

\section{本相假說}

\begin{enumerate}
\item{觀測原理}
\item{群性不滅}
\end{enumerate}

\section{觀測原理}

觀測原理({\it Observation Principle}),

\subsection{波}


\subsection{位置空間波函數}

\begin{equation}
P_{a \leq x \leq b} = \int_{a}^{b} |\Psi(x,t)|^2 \partial{x}
\end{equation}


\subsection{動量空間波函數}


\begin{equation}
P_{a \leq p \leq b} = \int_{a}^{b} |\Psi(p,t)|^2 \partial{p}
\end{equation}



粒子

\begin{equation}
\nu = \frac{E}{h}
\end{equation}

\begin{equation}
\lambda = \frac{h}{p}
\end{equation}



本相態:物理實體無法彼此交互作用的狀態


\section{不確定性原理}

\begin{equation}
\Delta x \Delta p \geq \frac{\hbar}{2}
\end{equation}

\section{本相理論}

Oriphase Theory

物理實體彼此不進行交互作用時,呈現「本相態」,既非粒子、亦非波,粒子與波的狀態必須在有「觀測者」出現的條件下才會呈現,因而導致「不確定性原理」的展現。


本相$\rightleftharpoons$粒子


粒子$\rightarrow$本相












本相$\rightarrow$粒子






\section{粒子之間的交互作用}

\begin{center}
\includegraphics[width=5.00in,height=3.74in]{particles.png}
\\
\includegraphics[width=5.00in,height=4.74in]{elementary.png}
\end{center}

強作用力


弱作用力


電磁作用力





波$\rightleftharpoons$本相



粒子$\rightarrow$本相








本相$\rightarrow$粒子
































\section{本相群特徵}

一、無觀測者
任何觀測者加入量子系統當中,都會導致「本相態」發生相變,形成「波」、「粒子」或 其他形式的物理實體,因此「本相態」無法被直接觀測。「本相態」存在的必要條件即為「無觀測受體」。

「本相態」具備一種「孤立條件」,在「孤立條件」成立之下,它不會發生相變,因而無法被探測到。「孤立條件」被破壞時,則發生物理實體相變,呈現「波」、「粒子」或粒子衰變形式。本相態的「孤立條件」基本上與「不確定性原理」在物理意象上等價。

二、展現群性
「本相態」通常以群出現,絕少有「孤立本相」,群的本體特徵可以被展現出來,例如,總能量不變或佔據特定量子空間,這些群性不會消滅掉。

三、群性可間接觀測
由於「本相態」群性不滅,因而可以透過間接的方式來間接觀測其實際存在性。



四、波粒疊加態




五、信息攜帶




實驗設計





\section{里德伯常量}

里德伯常量,{\it Rydberg constant}。

\begin{equation}
R_{\infty} = 1.0973731568539(55) \times 10^{7} m^{-1}
\end{equation}

\begin{equation}
R_{\infty}  = \frac{m_{e}e^{4}}{8\epsilon_{0}^{2}h^{3}c} = \frac{2\pi^{2}m_{e}e^{4}}{(4\pi\epsilon_{0})^{2}h^{3}c} 
\end{equation}


$m_{e}$電子質量
\\
e電荷
\\
$\epsilon_{0}$真空電容率
\\
c光速
\\
h普朗克常數
\\


\begin{equation}
R_{A} = \frac{R_{\infty}}{1+\frac{m_{e}}{M}}
\end{equation}






























\section{時間量子模型}

時間量子模型,{\it Caldirola-Montaldi Chronon Model}。

\begin{equation}
\theta_{0}m_{0} = \frac{e^{2}}{6\pi\epsilon_{0}c^{3}}
\end{equation}

$\theta_{0}$為時間量子












































\section{CKM矩陣}

CKM矩陣,{\it Cabibbo-Kobayashi-Maskawa},卡比博-小林誠-益川敏英矩陣。
\\
\begin{equation}
\left[
\begin{matrix}
\cos{\theta_{1}}
&
-\sin{\theta_{1}}\cos{\theta_{3}}
&
-\sin{\theta_{1}}\sin{\theta_{3}}
\\
\sin{\theta_{1}}\cos{\theta_{2}}
&
\cos{\theta_{1}}\cos{\theta_{2}}\cos{\theta_{3}} - 
\sin{\theta_{2}}\sin{\theta_{3}}e^{i\delta}
&
\cos{\theta_{1}}\cos{\theta_{2}}\sin{\theta_{3}} + 
\sin{\theta_{2}}\cos{\theta_{3}}e^{i\delta}
\\
\sin{\theta_{1}}\sin{\theta_{2}}
&
\cos{\theta_{1}}\sin{\theta_{2}}\cos{\theta_{3}} + 
\cos{\theta_{2}}\sin{\theta_{3}}e^{i\delta}
&
\cos{\theta_{1}}\sin{\theta_{2}}\sin{\theta_{3}} - 
\cos{\theta_{2}}\cos{\theta_{3}}e^{i\delta}
\end{matrix}
\right]
\end{equation}
\\
夸克交聯函數:$ \begin{pmatrix} \theta_{12} & \theta_{13} & \theta_{23} & \delta_{13} \end{pmatrix} $
\\
\begin{equation}
\left[
\begin{matrix}
V_{ud} & V_{us} & V_{ub} \\
V_{cd} & V_{cs} & V_{cb} \\
V_{td} & V_{ts} & V_{tb}
\end{matrix}
\right]
=
\left[
\begin{matrix}
c_{12}c_{13}
&
s_{12}c_{13}
&
s_{13}e^{-i\delta_{13}}
\\
-s_{12}c_{23}-c_{12}s_{23}s_{13}e^{i\delta_{13}}
&
c_{12}c_{23}-s_{12}s_{23}s_{13}e^{i\delta_{13}}
&
s_{23}c_{13}
\\
s_{12}s_{23}-c_{12}c_{23}s_{13}e^{i\delta_{13}}
&
-c_{12}s_{23}-s_{12}c_{23}s_{13}e^{i\delta_{13}}
&
c_{23}c_{13}
\end{matrix}
\right]
\end{equation}
\\
$
\theta_{12} = 13.04 \pm 0.05 ^\circ
\\
\theta_{13} = 0.201 \pm 0.011 ^\circ
\\
\theta_{23} = 2.380 \pm 0.06 ^\circ
\\
\delta_{13} = 1.200 \pm 0.08
\\
\\
\theta_{13} \ll \theta_{23} \ll \theta_{12} \ll 1
$
\\
\\
Jarlskog
\\
$ V_{ud}V_{ub}^{*} + V_{cd}V_{cb}^{*} + V_{td}V_{tb}^{*} = 0 $
\\
么正三角形:(0,0),(1,0),($\overline{\rho}$,$\overline{\eta}$)
\\
($\overline{\rho}$,$\overline{\eta}$)為CP破壞不變數。
\\
\\
夸克波函數


















\section{PMNS矩陣}

Pontecorvo–Maki–Nakagawa–Sakata Matrix,龐蒂科夫-牧二郎-中川昌美-坂田昌一矩陣。
\\
\begin{equation}
\left[
\begin{matrix}
\nu_{e}
\\
\nu_{\mu}
\\
\nu_{\tau}
\end{matrix}
\right]
= 
PMNS
\cdot
\left[
\begin{matrix}
\nu_{1}
\\
\nu_{2}
\\
\nu_{3}
\end{matrix}
\right]
\end{equation}
\\
\begin{equation}
PMNS = 
\left[
\begin{matrix}
U_{e 1} & U_{e 2} & U_{e 3} \\
U_{\mu 1} & U_{\mu 2} & U_{\mu 3} \\
U_{\tau 1} & U_{\tau 2} & U_{\tau 3}
\end{matrix}
\right]
\end{equation}
\\
中微子振蕩現象
\\
\begin{equation}
PMNS = 
\left[
\begin{matrix}
\cos{\theta_{12}}\cos{\theta_{13}}
&
\sin{\theta_{12}}\cos{\theta_{13}}
&
\sin{\theta_{13}}e^{-i\delta}
\\
-\sin{\theta_{12}}\cos{\theta_{23}} - 
\cos{\theta_{12}}\sin{\theta_{23}}\sin{\theta_{13}}e^{i\delta}
&
\cos{\theta_{12}}\cos{\theta_{23}} - 
\sin{\theta_{12}}\sin{\theta_{23}}\sin{\theta_{13}}e^{i\delta}
&
\sin{\theta_{23}}\cos{\theta_{13}}
\\
\sin{\theta_{12}}\sin{\theta_{23}} - 
\cos{\theta_{12}}\cos{\theta_{23}}\sin{\theta_{13}}e^{i\delta}
&
-\cos{\theta_{12}}\sin{\theta_{23}} - 
\sin{\theta_{12}}\cos{\theta_{23}}\sin{\theta_{13}}e^{i\delta}
&
\cos{\theta_{23}}\cos{\theta_{13}}
\end{matrix}
\right]
\end{equation}
\\
$
\theta_{12} \thickapprox 45^\circ
\\
\theta_{23} \thickapprox 34^\circ
\\
\theta_{13} \thickapprox 4.4^\circ
$
\\
\begin{equation}
PMNS = 
\left[
\begin{matrix}
0.85 & 0.53 & 0.00 \\
-0.37 & 0.60 & 0.71 \\
0.37 & -0.60 & 0.71
\end{matrix}
\right]
\end{equation}
\\































\section{波、本相及粒子的物理意象}
\begin{center}
「本相態」如同麵粉
\\
\includegraphics[width=2.00in,height=1.33in]{flour.png}
\end{center}

一團麵粉加上一團麵粉,依然是一團麵粉。
\\
\begin{tabular}{ccccc}
\includegraphics[width=1.50in,height=1.00in]{flour.png}
&
加上
&
\includegraphics[width=1.50in,height=1.00in]{flour.png}
&
依然是
&
\includegraphics[width=2.00in,height=1.33in]{flour.png}
\end{tabular}
 \\
 \\
新的一團麵粉看起來比較大,僅有本相群性被展現表達出來。
「大小」,即為「本相群性」的一種。
\\
 \\
\begin{center}
「粒子態」如同麵團
\\
\begin{tabular}{ccc}
\includegraphics[width=2.00in,height=1.33in]{flour.png}
&
$\Longrightarrow$
&
\includegraphics[width=2.00in,height=1.31in]{dough.png}
\end{tabular}
\\
\end{center}
合適的條件下,例如具備「觀測者」存在的狀況下,「本相態」與「粒子態」之間可以產生相變,在「觀測者」的「感受」下,它是一個物理實體的存在。
\begin{center}
「波相」如同手拉麵
\\
\begin{tabular}{ccc}
\includegraphics[width=2.00in,height=1.33in]{flour.png}
&
$\Longrightarrow$
&
\includegraphics[width=2.00in,height=1.41in]{ramen.png}
\end{tabular}
\\
\end{center}
「物理實體」在本相態的狀況下,無法直接被觀測,但產生相變以後,可以用多種方式呈現,例如:「波」、「粒子」及「波粒疊加態」。
\\
\section{本相孤立條件}



\section{觀察者效應}

Observer Effect

「本相」(Oriphase)與「觀測受體」(Observer Acceptor)由於會產生交互作用,導致「本相」產生相變,也就是當「物理實體」成為「觀測受體」,兩者在「孤立條件」被打破的條件下,「觀測受體」將導致「本相」相變為波或粒子等一類的「物理實體」。此時,兩個「物理實體」依循交互作用發生物理現象。

也就是說「觀測受體」的存在是將「本相」實體化的重要交互作用力,不存在「觀測受體」時,物理實體處於非波亦非粒子的狀態。

黑洞物質即為本相群
古典黑洞當中,由於存在隔離「觀測受體」的屏蔽,「孤立條件」無法被打破,因此,黑洞物理實體是以「本相群」展現,因而黑洞無法被直接觀測,但其群性可以被間接觀測。在「本相假說」當中,黑洞物質即為「本相群」,是一種物理實體,而非真的有什麼洞。



\section{海森堡不確定性原理}

Uncertainty Principle
\\
\begin{large}
$$ \Delta x \Delta p \geq \frac{\hbar}{2} = 5.272859 \times 10^{-35} $$
\end{large}
\\

物理實體一旦處於無觀測者的「孤立條件」下,就會處於本相態,決定本相「孤立條件」的重要關鍵,即為「海森堡不確定性原理」。

「海森堡不確定性原理」是「粒子態」向「本相態」發生相變的孤立條件。

海森堡不確定性原理:粒子態$\rightarrow$本相態

\section{本相孤立條件}

一旦處於孤立條件,本相態就會佔據優勢,物理實體傾向於以本相態呈現。要取得關鍵的孤立條件,物質之間的交互作用與海森堡不確定性原理是判斷依準。




\section{卡西米爾效應}

Casimir effect

真空能量

\begin{large}
$$ E = \frac{\hbar\omega}{2} $$
\end{large}

目前測得的零点能量,實際上是「本相能」。




\section{霍爾效應}

Hall effect

霍爾效應(Hall effect)是指當固體導體有電流通過,且放置在一個磁場內,導體內的電荷載子受到洛倫茲力而偏向一邊,繼而產生電壓。電壓所引致的電場力會平衡洛倫茲力。

除導體外,半導體也能產生霍爾效應,而且半導體的霍爾效應要強於導體。

在導體上外加與電流方向垂直的磁場,會使得導線中的電子受到洛倫茲力而聚集,從而在電子聚集的方向上產生一個電場,此一電場將會使後來的電子受到電力作用而平衡掉磁場造成的洛倫茲力,使得後來的電子能順利通過不會偏移,此稱為霍爾效應。而產生的內建電壓稱為霍爾電壓。

方便起見,假設導體為一個長方體,長度分別為L、b、d ,磁場垂直於Lb平面。當電流經過Ld平面的方向,電流I=nv(Ld) ,而n為電荷密度。設霍爾電壓為,導體沿霍爾電壓方向的電場為 。設磁感應強度為B 。












\section{Odo物質密度分布}


\begin{large}
$$ \rho = \rho_{0} + \frac{k}{r^3} $$
\end{large}







































\section{T物質}

  1. 均匀的光(激光束中心处)加上厚(>5mm)单缝后,随着缝宽d变细透过的光照度I非线性变小I=k/d2。钢(无磁性的)和玻璃外无磁场电场,并且在空气或真空中磁场电场对光几乎无作用,所以称这种物质为T物质。
  2.用偏振光证明固体表面T物质有各向异性。
  3.用干涉实验证明T物质能改变光的相位差。
  4.用直边衍射证明T物质能偏转、反射光。
  5.用前人的气体折射率数值证明T物质存在于原子和电子中。以上实验重复多次任何人都能重现,排除上面实验现象是表面吸附气体产生的证据有二:
1.是用真空电子管中的面电极侧面偏转激光束,效果同空气中相同。
2.是日全食贝利珠的光偏转,月球表面是真空。
  开始学原子物理时习惯将电子想像为一个小球,后来学量子力学又认为核外电子象一片云,从高能物理实验知道在<0.1fm时仍没发现电子的结构,大都份人认为电子是一个点粒子。认为电子是点粒子有几个问题:
1.无法解释介质中光速为什么小于真空中光速问题,原子中除电子和核外就是真空,若认为光与电子散射是介质光速小的原因,其必然的结论是介质光速与介质厚度成反比关系,这与实验结果矛盾。
2.无法解释上面的四个实验。
  二十多年前从几个方面总结归纳比较,又凭直觉提出电子的T物质密度分布是.r是考察点到电子中心距离,r非常大时的T物质密度是ρ,ρ0也等于真空中的T物质密度,k是系数。用该公式能解释上面几个实验,该公式和过去的实验和数据是相容的。从公式能看到电子是中心密度高边缘密度低的物体,象一小团云。详见川大出版社新书<光子衍射与T物质>。



\section{阿瑞尼斯方程式}

Arrhenius equation



\section{最小作用量原理}
Least action principle



\section{通俗的物理詮釋}

目前只知的物理實體狀態,僅有「波」與「粒子」,基本粒子在目前的量子力學當中,靜止狀態是一種點狀不佔體積的物理實體。質子與中子由於屬於重子,係由三個夸克所組成,因此,其質子與中子是可以測量出其佔有空間的大小。而幾乎所有的基本粒子,目前都確定是小於$10^{-17}$公尺,真正的大小還無法測量出來。

目前世界上有許多座正負電子對撞機,在解釋正負電子對撞的過程當中,目前還存在許多全球所有物理學家無能力真正解釋清楚的問題。此外,光子偏振問題當中,長程光子傳播過程,會發生短暫光子轉變為正負電子對,並且在極短的時間內,互相湮滅轉換回光子的過程,此即稱為「光子偏振」。第三,在幾乎所有的物理實驗當中,都發現「真空基態能量」,比現有量子力學的預測相差數十個數量級($10^{45}$),這個數量級大到荒謬而無法正確解譯其物理涵義。這三個重要的物理現象,都發生嚴重的物理解釋問題,目前依然沒有任何一個學說有辦法解釋清楚。

「本相理論」在現有量子力學的框架下,增加一項極為重要的假設「觀測原理」,使得現有量子力學不需要使用「額外維度」及「虛粒子假說」,就可以輕易解釋清楚這些目前被視為高度疑難的物理本源性問題。

\subsection{觀測原理}

「本相理論」在現有量子力學的框架下,增加了一組正反假設:
\\
\\
正假設:粒子或波需要有觀測受體共處於系統中才會處於粒子或波的狀態下。
\\
\\
反假設:孤立條件下,物理實體無法被觀測到。
\\
\\
正反假設是否等價,目前還無法確認。因此,目前觀測原理是同時交錯使用這兩個假設,將其當成是同一個假設的一體兩面對待。
\\

\subsection{物理蘊含意義}

在量子力學當中增加這個假設,有許多重要的物理蘊含意義。
\\
\\
一、物理實體沒有觀測者存在時,或是處於「孤立條件」成立的條件下,不會進入「波」或「粒子」的狀態下,此時物理實體進入「本相態」,而運動中的「波」或「粒子」進入本相態時,其群性依然會展現,此時則為一種稱為「Odo物質」或是「空性體」的物理實體,是「本相態」的其中一種共振實體,這種物理實體,非波,亦非粒子。
\\

二、原有使用「虛粒子假說」解釋的物理現象,實際上是由於進入了「本相態」,因此,本相理論可以直接替換掉現有量子物理當中的「虛粒子假說」。據此理論,「波」與「粒子」實際上是不能直接與量子漲落有直接關連性,「波」與「粒子」僅能與「本相態」互相震盪,爾後「本相態」才與量子漲落產生直接的關連性。這為許多現有無法解釋清楚解釋的物理現象,找到了完美的真正物理解釋。
\\

三、原有的「真空基態能量」,實際上是「本相態能量」,直接解決了數量級差距的問題,而無需引入額外維度的解決方案。
\\

四、「本相理論」的「觀測原理」,直接描述處於「非波」及「非粒子」狀態下的物理實體,由於能量不滅定理,處於「本相態」的物理實體總能量是不會消失,必須透過群性展現。因此,「本相理論」直接涵蓋了現有的最新理論「非粒子理論」。同時,由於所描述的物理實體,無法為現有以粒子或波作為探測手段的方法所觀測到,因此,僅能採用新的「弱測量技術」來間接觀測「本相態群性展現」。
\\

五、「本相態群性展現」的特性直接解決了量子引力的問題,僅需在量子力學的框架下,即可解釋清楚量子長程力的問題,以現有理論而言,即為廣義相對論,本相理論本身處於量子力學的框架下,同時「群性展現」的部份亦可替代廣義相對論,使得更多的物理現象得到相對更妥善的物理詮釋。
\\

六、薛丁格的貓問題,在本相理論當中的假設,假定了「薛丁格的貓」確實是半死不活,判定「本相理論」的成立與否,直接決定這個量子物理經典問題的正確與否。「本相理論」成立,則物質存在中間態,也就是「薛丁格的貓」是半死不活。「本相理論」不成立,則物質不存在中間態,「薛丁格的貓」則非死即活,沒有含糊空間。而「本相理論」僅能使用「弱測量技術」來探測,這為薛丁格的貓問題,提供了一個確實存在可驗證的技術手段。
\\

七、使用本相理論,量子力學當中將基本粒子假設為點狀不佔體積的物理實體,必須修改為佔有一定量子空間區域的物理實體,而不是一個無限小的數值,這使得使用弱測量技術來探測基本粒子佔有空間大小,成為真實可能。
\\
這種物質的密度表達式為:$$ \rho = \rho_{0} + \frac{k}{r^3} $$。
\\

\subsection{利用觀測原理解釋現有物理現象}

正負電子對撞過程

舉正負電子對撞過程當中的一例:
\begin{equation}
e^+ + e^- \xrightarrow{\gamma} D^+ + D^-
\end{equation}
正負電子對撞以後,產生了$D^+$與$D^-$介子,就其過程而言,即為正電子與負電子對撞以後湮滅,先產生了正反魅夸克對(charm quark),其後產生了正反下夸克對(down quark),正魅夸克與反下夸克產生了$D^+$介子,反魅夸克與正下夸克產生了$D^-$介子,這是量子物理當中,最經典的由輕子(電子)轉換成夸克的典範。

然而這整個過程,爭議最大的一個部分是正負電子對撞湮滅,到正反魅夸克對與正反下夸克對產生的期間,使用了「虛粒子假說」,而「虛粒子假說」目前根本無法驗證。

改用「本相理論」解釋,由於正負電子對撞湮滅,系統中的交互觀測受體同時消失,因此正負電子全部都進入「本相態」,爾後由於受到加速器當中的探測器影響,「本相態」的物理實體必須以「波」或「粒子」的方式呈現,因此先後產生了「光子」($\gamma$射線)、「正魅夸克」、「反魅夸克」、「正下夸克」及「反下夸克」,然後再結合成$D^+$與$D^-$介子。「本相理論」不僅提供了完美的物理詮釋,也供應了透過實驗驗證的方法。
\\
\\
光子長程傳播偏振問題
\\
\\
目前的超新星觀測,發現了一個較為難以解釋的光子傳播時間問題。
\\
\\
由於超新星爆發時,中微子會先發出,爾後發出光子,中微子目前已知具有質量,而光子不具備質量,理論上而言,應當是不具備質量的光子先被觀測到。然而,目前對所有超新星爆發的觀測,全部都發現,光子一般都比具有質量的中微子晚抵達地球數個小時。
\\
\\
針對這個疑難,目前天體物理學提出了一個「光子長程傳播偏振」的解決方案。光子在長程傳播的過程當中,由於會發生光子轉變成正負電子的偏振過程,這個過程會賦予光子具備質量,使得光速的傳播過程減慢,導致光子實際抵達地球的時間比中微子慢。
\\
\\
然而,這個解釋方案不僅不能解釋光線在長程傳播當中,會發生「引力透鏡」的現象,也無法解釋長程傳播當中為何會發生光子偏振的物理現象。
\\
\\
而改用本相理論以後,則有極度簡單完美的解釋方案。
\\
\\
光子在長程傳播的過程當中,經常會穿過宇宙空洞區,宇宙空洞區通常就是什麼都沒有,這導致光子喪失「觀測受體」,進而使光子進入「本相態」而成為「空性體」,「空性體」會展現群性,使得原有的光子能量,以群性展現,因而在通過強引力區時,光子能量變成了物質質量,以群性與引力進行交互作用,因而產生光子的「引力透鏡」現象。如此即可直接排除使用廣義相對論的「引力磁性理論」,只要使用本相理論就可以修改成單純使用量子力學成功解釋「引力透鏡」現象。
\\
\\
當光子轉換成的「空性體」穿出宇宙空洞區或是又再度獲得「觀測受體」的觀測作用,「空性體」再度轉換回光子或是正負電子對,如此全部的觀測到的物理現象就得到與觀測數據吻合的完美結果。整個過程當中,完全不需要使用廣義相對論來進行解釋。
\\
\\
黑洞問題
\\
\\
如果假設「黑洞物質」即為本相理論當中的「本相態空性體」,由於「空性體」是一種處於「孤立條件」底下的物理實體,無法透過「波」或「粒子」的探測手段觀測到,但可以透過非粒子型態的長程力或是弱測量方法來探測到,例如引力波。
\\
\\
如此,黑洞無法被觀測,但具有強引力的本質,立刻得到物理本質性的正確解釋,即「黑洞」的物理實體是本相理論當中的本相態「空性體」,而非真的在太空中有個洞,如此黑洞邊界的「量子防火牆」亦同時得到完美解釋。
\\
\\
使用本相理論可以是用極度簡單的經典量子物理來完美解釋黑洞成功,而無需動用廣義相對論或額外額度。
\\
\\
傳統解決黑洞問題的物理數學手法通常要動用數十個未經檢驗的假設,而使用本相理論僅需檢驗「觀測原理假設」是否成立即可,這使得量子力學的完備化成為真正的可能。
\\
\\
暗能量與暗物質
\\
\\
目前對宇宙的觀測所得,COBE與普朗克衛星的觀測數據顯示,暗能量佔宇宙總質量的68.3\%,暗物質佔宇宙總質量的26.8\%,而正常物質則佔據4.9\%。目前對於暗能量與暗物質的物理實體組成,從第五元素(Quintessence)、$\Psi$粒子、強交互作用粒子(SIMP)及弱交互作用大粒子(WIMP)等等,共計數十種,沒有任何一個解釋成功。
\\
\\
使用本相理論卻可以單純使用極簡單的大一等級數學成功解釋。
\\
\\
我們的宇宙是主要是由介子與重子組成的,基本上是二夸克與三夸克物質。二夸克與三夸克物質的量子色動力學組合狀態,介子有540種,重子則有8562種,合計為9102種。量子色動力學成功的束縛態,介子有108種,重子則有432種,共計540種。其中,由於頂夸克重量太重,正常無法成為束縛態,會在極度短的時間內衰變成其他粒子,因此,也無法被正常的手段探測到。於是,540種束縛態,扣除頂夸克的組合,再扣除已經確定的四個極度難以成功束縛的底夸克組合,則餘下446種。
\\
\\
$\frac{446}{9102} = 4.900022\%$
\\
\\
也就是說,在9102種夸克組合當中,僅有446種束縛態成為粒子型態,並且可以被我們所觀測到,也就是佔據整個宇宙的總質量剛好與觀測吻合,為$4.900022\%$,加上其他基本粒子的觀測值,與觀測數據高度吻合,誤差達到0.001\%,為$5\delta$,而這個數值絕對並非巧合。
\\
\\
9102種夸克組合扣除446種剩下的8656種夸克組合,都處於非束縛態,這些非束縛態處於非波、非粒子的狀態,無法被現有的物理技術直接觀測到,因為它們都處於「本相態」,它們實際存在,但無法被直接觀測到,僅能採用弱測量技術進行間接觀測。
\\
\\
採用本相理論,使用僅一個假設,得到極度簡單而合理的物理解釋。相對於其他理論,使用大量未經驗證的假設,甚至必須使用未經驗證的額外維度來趨近觀測數據,本相理論的解釋極度簡單而合理,並且實際可以有驗證手段來證實假說是否正確。
\\
\\
鬼魂現象
\\
\\
使用本相理論五分鐘快速解釋鬼魂現象。
\\
\\
由於本相理論預測95.1\%的物理實體,都無法使用「波」或「粒子」的方式觀測。而非束縛態的物理實體,的確在量子色動力學當中的列表佔據大部分,加上「觀測原理」僅限制「波」與「粒子」無法觀測到「本相態」,並未排除「本相態」可以觀測到「波」與「粒子」的可能性。因此使用本相理論推測鬼魂現象,人在生前若是意志力強大,精氣神豐滿,或是不知自己已經死亡,極有可能在死亡時,人的本相態離開肉體,而繼續保持本相態物理實體之間的交互作用極度完整,因而形成「神」或「鬼」,也就是傳統所說的「魂魄」就是以本相態存在。
\\
\\
人若死時,功德圓滿,受眾人祝福,則原有的本相態,也就是俗稱的「靈體」,充滿祝福的本相能量,具備一定程度進入束縛態實體化的能力,使得活人在某種程度上,可以感覺到「靈體」的存在,因而被稱之為「神」。
\\
\\
反之人若死時,怨氣很深,或是招眾人詛咒,則原有的本相態「靈體」,充滿惡毒的本相能量,陰魂久之不散,因而被稱之為「鬼」。
\\
\\
而傳統民間的「超渡」,實際的物理作用,就在於分解本相態物理實體之間的交互作用,瓦解「靈體」維繫完整性的能量,讓能量重新回歸融入大自然。
\\
\\
「神」與「鬼」的物理學具體探測手法,便可使用「弱測量方法」來探測,若是技術進一步推進,則長期維持「靈體」不滅,以及與「靈體」之間使用「弱測量方法」來建立溝通管道,是可以有具體的物理學實驗方法來實現。


\end{document} 
